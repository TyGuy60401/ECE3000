\documentclass[12pt]{article}

%
%Margin - 1 inch on all sides
%
\usepackage{hyperref}
\usepackage[letterpaper]{geometry}
\usepackage{times}
\geometry{top=1.0in, bottom=1.0in, left=1.0in, right=1.0in}

%
%Doublespacing
%
\usepackage{setspace}
\doublespacing

%
%Rotating tables (e.g. sideways when too long)
%
\usepackage{rotating}


%
%Fancy-header package to modify header/page numbering (insert last name)
%
\usepackage{fancyhdr}
\pagestyle{fancy}
\lhead{} 
\chead{} 
\rhead{Davis \thepage} 
\lfoot{} 
\cfoot{} 
\rfoot{} 
\renewcommand{\headrulewidth}{0pt} 
\renewcommand{\footrulewidth}{0pt} 
\setlength{\headheight}{14.5pt}
%To make sure we actually have header 0.5in away from top edge
%12pt is one-sixth of an inch. Subtract this from 0.5in to get headsep value
\setlength\headsep{0.333in}

%
%Works cited environment
%(to start, use \begin{workscited...}, each entry preceded by \bibent)
% - from Ryan Alcock's MLA style file
%
% \newcommand{\bibent}{\noindent \hangindent 40pt}
% \newenvironment{workscited}{\newpage \begin{center} Works Cited \end{center}}{\newpage }

%
%Bibtex and Bibliography

\usepackage[american]{babel}
\usepackage{csquotes}
\usepackage[style=mla,backend=biber]{biblatex}
\addbibresource{refs.bib}


%
%Begin document
%
\begin{document}
\begin{flushleft}

%%%%First page name, class, etc
Ty Davis\\
Dr. Eric Gibbons\\
ECE 3000\\
25 November 2023\\


%%%%Title
\begin{center}
Is Moore's Law Really Dead
\end{center}



%%%%Changes paragraph indentation to 0.5in
\setlength{\parindent}{0.5in}
%%%%Begin body of paper here

The principle behind Moore's law was first introduced
in 1965, and since then it has driven innovation in the
electrical engineering industry. This principle asserts
that the performance of leading computer processors
doubles roughly every one and a half years and it was 
proposed originally by Gordon E. Moore. In his 1965 paper
he talked about the decreasing space between, size of, and
cost of transistors found on integrated circuits. He
stated, "With unit cost falling as the number of components
per circuit rises, by 1975 economics may dictate squeezing
as many as 65,000 components on a single silicon chip"
\parencite{4785860}. A number in the range of 65,000
may have been a bold claim in 1965, but the recent
flagship processor from Intel -- the core i9 13900k
-- boasts up to 26 billion transistors on a single
chip. This dramatic increase, albeit spanning several
decades, shows that the original claims from Moore have
been able to hold true since its inception. Despite the
constant growth that has been prevalent over many years,
Moore's law has seemed to slow down more and more in 
the recent years. As components get smaller and smaller,
we encounter physical boundaries and roadblocks which 
present many difficulties in the manufacturing processes
of newer and faster chips.



\newpage


\newpage
\nocite{*}
\printbibliography

\end{flushleft}
\end{document}