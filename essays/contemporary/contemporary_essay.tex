\documentclass[12pt]{article}

%
%Margin - 1 inch on all sides
%
\usepackage{hyperref}
\usepackage[letterpaper]{geometry}
\usepackage{times}
\geometry{top=1.0in, bottom=1.0in, left=1.0in, right=1.0in}

%
%Doublespacing
%
\usepackage{setspace}
\onehalfspacing

%
%Rotating tables (e.g. sideways when too long)
%
\usepackage{rotating}


%
%Fancy-header package to modify header/page numbering (insert last name)
%
\usepackage{fancyhdr}
\pagestyle{fancy}
\lhead{} 
\chead{} 
\rhead{Davis \thepage} 
\lfoot{} 
\cfoot{} 
\rfoot{} 
\renewcommand{\headrulewidth}{0pt} 
\renewcommand{\footrulewidth}{0pt} 
\setlength{\headheight}{14.5pt}
%To make sure we actually have header 0.5in away from top edge
%12pt is one-sixth of an inch. Subtract this from 0.5in to get headsep value
\setlength\headsep{0.333in}

%
%Bibtex and Bibliography
%
\usepackage[american]{babel}
\usepackage{csquotes}
\usepackage[style=mla,backend=biber]{biblatex}
\addbibresource{refs.bib}


%
%Begin document
%
\begin{document}
\begin{flushleft}

%%%%First page name, class, etc
Ty Davis\\
Dr. Eric Gibbons\\
ECE 3000\\
25 November 2023\\


%%%%Title
\begin{center}
Is Moore's Law Really Dead?
\end{center}



%%%%Changes paragraph indentation to 0.5in
\setlength{\parindent}{0.5in}
%%%%Begin body of paper here

The principle behind Moore's law was first introduced
in 1965, and since then it has driven innovation in
the electrical engineering industry. This principle
asserts that the performance of leading computer processors
doubles roughly every one and a half years and it was
proposed originally by Gordon E. Moore. In his 1965
paper he talked about the decreasing space between,
size of, and cost of transistors found on integrated
circuits. He stated, "With unit cost falling as the
number of components per circuit rises, by 1975 economics
may dictate squeezing as many as 65,000 components
on a single silicon chip" \parencite{4785860}. A number
in the range of 65,000 may have been a bold claim in
1965, but the recent flagship processor from Intel
-- the core i9 13900k -- boasts up to 26 billion transistors
on a single chip. This dramatic increase, albeit spanning
several decades, shows that the original claims from
Moore have been able to hold true since its inception.
Despite the constant growth that has been prevalent
over many years, Moore's law has seemed to slow down
more and more in the recent years. As components get
smaller and smaller, we encounter physical boundaries
and roadblocks which present many difficulties in
the manufacturing processes of newer and faster chips.

With chips being manufactured in the nanometer scale,
we encounter some new problems that have never been
encountered before. Findings from an article showed
that "Further reduction in insulator thickness would
have resulted in unacceptable (and exponential) increases
in gate leakage current through direct quantum tunneling"
\parencite{7878935}. Direct quantum tunneling of electrons
through oxide thickness is a phenomenon that shows
that even if advances are made in finding processes
to allow the manufacturing of smaller and smaller components,
other physical limitations will likely arise which
can prevent the functionality of ever-smaller components.
Another physical challenge that is presented with
minute components is the difficulty to discern between
activated and deactivated currents. From the same
article we learn that, "Further reduction in operating
voltage swing would have resulted in either unacceptably
low channel current in the “on” state (unacceptable
decreases in switching speed) or increased leakage
current in the “off” state (unacceptable increases
in passive power)" \parencite{7878935}. With these
presented difficulties, in order for a future continuation
of Moore's law, the future advances in manufacturing
technology will have to present unique solutions.
Until this point, the increases in computing power
have come from placing more, smaller transistors on
the same size chip. Due to these difficulties, solutions
will have to come from unique innovations.

At the present time, the most-explored solution for
this problem is to expand the design of integrated
circuits to the third dimension. In an article by Samuel
K. Moore, we find a few examples of 3D integrated circuit
design that are in practice today. In the consumer
space, the most notable of the examples is the use
of 3D V-Cache in AMD's most recent consumer processors.
By stacking cache modules directly on top the other
components in their top-of-the-line gaming/productivity
CPU's they were able to achieve an increase of performance
of up to 15 percent. Another notable example of 3D
chip design being used today is Intel's Ponte Vecchio
Supercomputer chip, which was also highlighted in
this same article. "Using both 2.5D and 3D technologies,
Intel squeezed 3,100 square millimeters of silicon—nearly
equal to four Nvidia A100 GPUs—into a 2,330-mm$^2$ footprint"
\parencite{9792148}. These innovations allowed Intel's
supercomputer chip to finally "pierce the exaflop barrier"
\parencite{9792148}.

Techniques for 3D chip design expand beyond stacking
cache on top of other components. The pioneers of 3D
integrated circuit design have found ways to stack
CMOS components directly atop one another. Researchers
were able to find a configuration of CMOS components
that allowed the design of an inverter that had a footprint
half the size of a typical inverter designed by
placing components side-by-side. "Combined with appropriate
interconnects, the 3D-stacked CMOS approach effectively
cuts the inverter footprint in half, doubling the
area density and further pushing the limits of Moore's
Law" \parencite{9976473}. Their research in the area has
led to several advances in the development of 3D silicon
chips, certainly beyond vertical inverter configurations.
They've successfully made entire integrated circuits
featuring 3D circuit constructions, stating, "We've made
wafers where the smallest distance between two sets
of stacked devices is only 55 nm. While the device
performance results we achieved are not records in
and of themselves, they do compare well with individual
nonstacked control devices built on the same wafer
with the same processing" \parencite{9976473}.

New innovations and solutions in the realm of processor
design are being presented constantly, and if we're
to expect a continuation in Moore's law and computation
performance increases in the future, the solutions
that are presented will need to take a different shape
from the solutions that have been presented through
the first 40 years of the existence of Moore's law
as a concept. Further innovations in the realm of newer
magnetic memory, computation techniques, and overall
rethinking of modern computation technologies are being
considered and studied each day. With the increase
of technologies and devices that require more demanding
computation power, the development of more efficient
and computationally effect devices will surmise, but
whether it will keep up with the development claims
of Moore's law will only be told by the future.



\newpage


\newpage
\nocite{*}
\printbibliography

\end{flushleft}
\end{document}